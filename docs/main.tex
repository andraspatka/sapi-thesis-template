\documentclass[12pt, twosides]{report}

\usepackage{graphicx}
\graphicspath{ {figures/} }
\usepackage[left=2.5cm,right=2cm,top=2.5cm]{geometry}


\usepackage[utf8]{inputenc}
\usepackage{t1enc}
\usepackage[magyar]{babel}
\linespread{1.5}

\usepackage{footnote}
\usepackage{subfigure}
\usepackage{float}
\usepackage[]{algorithm2e}
\usepackage{amsmath}
\usepackage{mathptmx}
\usepackage{pdfpages}

\usepackage{xcolor}
\usepackage{hyperref}
\hypersetup{
    colorlinks,
    linkcolor=black,
    citecolor=black,
	urlcolor={blue!80!black},
	unicode=true
}
% 
\urlstyle{same}

\usepackage{listings}
\definecolor{dkgreen}{rgb}{0,0.6,0}
\definecolor{gray}{rgb}{0.5,0.5,0.5}
\definecolor{mauve}{rgb}{0.58,0,0.82}
\definecolor{light-gray}{gray}{0.25}

\lstdefinestyle{yaml}{	
	backgroundcolor=\color{white}, % choose the background color;
	basicstyle=\fontsize{8}{8}\ttfamily,% the size of the fonts that are used for the code
	breakatwhitespace=false, % sets if automatic breaks should only happen at whitespace
	breaklines=true, % sets automatic line breaking
	commentstyle=\color{dkgreen},  % comment style
	deletekeywords={...},  % if you want to delete keywords from the given language
	escapeinside={\%}{)},  % if you want to add LaTeX within your code
	extendedchars=true,  % lets you use non-ASCII characters; for 8-bits encodings only, does not work with UTF-8
	frame=none,	 	 % adds a frame around the code
	keepspaces=true, % keeps spaces in text, useful for keeping indentation of code (possibly needs columns=flexible)
	keywordstyle=\color{blue}\bfseries, % keyword style
	otherkeywords={*,...}, % if you want to add more keywords to the set
	numbers=none,  % where to put the line-numbers; possible values are (none, left, right)
	numbersep=5pt, % how far the line-numbers are from the code
	numberstyle=\tiny\color{gray}, % the style that is used for the line-numbers
	rulecolor=\color{black}, % if not set, the frame-color may be changed on line-breaks within not-black text (e.g. comments (green here))
	showspaces=false,% show spaces everywhere adding particular underscores; it overrides 'showstringspaces'
	showstringspaces=false,  % underline spaces within strings only
	showtabs=false,  % show tabs within strings adding particular underscores
	%stepnumber=1,  % the step between two line-numbers. If it's 1, each line will be numbered
	stringstyle=\color{mauve}, % string literal style
	tabsize=2,
	columns=fullflexible  % Using fixed column width (for e.g. nice alignment)
	sensitive = true,
	morekeywords={name, runs-on, on, jobs, build, run, steps, uses}
}
\lstset{style=yaml}

\title{
	{Szimulált robotraj vezérlése}\\
	{\large Sapientia\\
	Erdélyi Magyar Tudományegyetem, Marosvásárhely}
}
\author{
	Patka, Zsolt-András\\
	\texttt{patka.zsolt-andras@student.ms.sapientia.ro}
	\and
	Szántó, Zoltán\\
	\texttt{zoltan.szanto@ms.sapientia.ro}	
}
\date{2020}

%%%%%%%%%%%%%%%%%%%%%%%%%%%%%%%%%%%%%%%%%%%%%%%%%%%%%%%%%%%%%%%%%%%%%%%%%
\begin{document}

\includepdf[pages={1,2,3}]{pdfs/allamvizsga_borito.pdf}

\section*{Extras}
\input{chapters/abstractRo}
\pagebreak

\includepdf[pages={6}]{pdfs/allamvizsga_borito.pdf}

\section*{Kivonat}
\input{chapters/abstractHu}
\pagebreak

\section*{Abstract}
\input{chapters/abstractEn}
\pagebreak

\pagenumbering{gobble}

\tableofcontents

\listoffigures

\chapter{Bevezető}
\pagenumbering{arabic}
\section{Cím}

Általános bevezető szöveg. A \ref{fig:child_dragged} ábrán látható ahogy egy robotraj együttesen elmozdít egy kislányt.

\begin{figure}[h]
    \centering
    \includegraphics[scale=0.6]{figures/images/literature/child_dragged_robots.png}
    \caption{Rövid szöveg a képről, hivatkozás \cite{parker2016multiple}}
    \label{fig:child_dragged}
\end{figure}

\section{Cím 2}

Két ábra egymás mellett (lásd \ref{fig:insbots} ábra).

\begin{figure}[h]
    \centering
    \hfill
    \subfigure[Insbot \cite{colot2004insbot}]{\includegraphics[scale=0.3]{figures/images/literature/insbot.png}}
    \hfill
    \subfigure[Insbot és csótányok interakciója \cite{garnier2011ants}.]{\includegraphics[scale=0.25]{figures/images/literature/insbot_cockroach.png}}
    
    \caption{Insbot és csótányok interakciója. Az insbot-ok képesek a csótányokat csalogatni.}
    \label{fig:insbots}
\end{figure}

\chapter{Szakirodalom áttekintése}
\section{Cím 1}

\subsection{Alcím 1}

Táblázat: 

\begin{figure}[h]
    \begin{table}[H]
    \begin{footnotesize}
        \begin{center}
            \begin{tabular}{p{2.5cm}|p{3.5cm}|p{4cm}|p{4cm}}
            \textbf{Kritériumok}                    & V\-rep                              & ARGoS  & Gazebo \\
            \hline         
            Ingyenes                            & Igen, van fizetős verzió is         & Igen             & Igen \\
            \hline         
            Absztrakciós szint                  & Valósághű         & Emelkedett absztrakciós szintet ajánl             & Valósághű  \\
            \hline         
            Robotrajokra optimalizált           & Nem optimalizált         & Teljesen optimalizált           & Képes, nagyobb erőforrásigény, mint az ARGOS-nak  \\
            \hline         
            Nyílt forráskódú                    & Igen         & Igen            & Igen  \\
            \hline         
            Támogatott programozási nyelvek     & C/C++, Python, Java, Lua, Matlab, Octave          & C/C++ és Lua             & C/C++  \\
            \hline         
            Valós robotok modelljei             & Igen          & Igen             & Igen  \\
            \end{tabular}
        \end{center}
    \end{footnotesize}
    \end{table}
    \label{table:simulators}
    \caption{V-REP, ARGoS, Gazebo összehasonlítása}
\end{figure}

Hivatkozás a táblázatra: \ref{table:simulators}

\chapter{Elméleti áttekintés}
\section{Elméleti áttekintés}

Pszeudokód: 

\begin{algorithm}[h]
    \begin{small}
    \KwData{Tanulási tényező ($\alpha \in (0, 1])$), $\epsilon > 0$}
    Véletlenszerű érték minden $Q_{1}(s,a)$ és $Q_{2}(s,a)$-nek, kivéve $Q($terminális$, \cdot ) = 0$, $s \in S$, $a \in A$ \;
    \For{minden epizód}{
        S inicalizálása\;
        \Repeat{S terminális állapot}{
        $A \leftarrow$ cselekvés, $S$ állapotban $\epsilon$-greedy szerint $Q_{1}+Q_{2}$ \;
        $A$ cselekedet végrehajtása, R és S' megfigyelése\;
        \eIf{ $50\%$ eséllyel}
            {$Q_{1}(S, A) \leftarrow Q_{1}(S, A) + \alpha [R + \gamma Q_{2}(S', arg\max_{a}Q_{1}(S', a)) - Q_{1}(S, A)]$}
            {$Q_{2}(S, A) \leftarrow Q_{2}(S, A) + \alpha [R + \gamma Q_{1}(S', arg\max_{a}Q_{2}(S', a)) - Q_{2}(S, A)]$}
        $S \leftarrow S'$\;
        }
    }
    \end{small}
    \caption{Dupla Q-tanulás \cite{sutton2018reinforcement}.}        
    \label{algo:double_q}
\end{algorithm}


Hivatkozás pszeudokódra: \ref{algo:double_q}.

\chapter{Rendszer specifikációi}
\section{Cím 1}

%Terv
\chapter{Gyakorlati megvalósítás} \label{chpt:implementation}
\section{Ágensek vezérlése}

\textbf{Hivatkozásra példa}

Az ágensek vezérléséhez a potenciálmező navigációs módszer volt felhasználva. Ez egy bevált módszer a robotrajok vezérléséhez \cite{szanto2015investigation}. Az alapötlete, hogy az akadályok taszító erővel hatnak az ágensre és a cél vonzó erővel. Ennek a két erőnek az eredője határozza meg az irányt amerre érdemes haladni.

\subsection{Potenciálmező navigáció}

\textbf{Egyenletekre példa}

A potenciálmező navigációs módszernél az erők nagysága az \eqref{eq:poti} egyenlet szerint van kiszámolva.

\begin{equation}
    \left\{
    \begin{array}{l}
        |\vec{f}_{push}| = a e ^ {- \frac{(x - b_{push}) ^ {2}}{2 c_{push}^2 }} \\
        |\vec{f}_{pull}| = a e ^ {- \frac{(x - b_{pull}) ^ {2}}{2 c_{pull}^2 }} \\
    \end{array}
    \right.
    \label{eq:poti}
\end{equation}

\begin{itemize}
    \item a: Gauss görbe magassága
    \item b: Gauss görbe középpontja
    \item c: Gauss görbe szélessége
\end{itemize}

\begin{equation}
    \vec{f}_{robot} = \sum_{i} \vec{f}_{push_{i}} + \sum_{i} \vec{f}_{pull_{i}}
    \label{eq:poti_eredo}
\end{equation}

Az eredő vektor a \eqref{eq:poti_eredo} képlet szerint volt kiszámolva. 


\chapter{Eredmények}
\section{Cím 1}

Eredmények leírása

\begin{figure}[h]
    \centering
    \includegraphics[scale=0.8]{figures/images/results/swarm_second_version.png}
    \caption{30 követő ágens, egy vezér}
    \label{fig:result_30_foll_1_lead}
\end{figure}





%következtetés
\chapter{Összefoglalás}
\input{chapters/summary}

\addcontentsline{toc}{chapter}{Irodalomjegyzék}
\bibliographystyle{ieeetr}
\bibliography{References}

%appendix
\appendix
\chapter{Függelék}
\section{Alfejezet}

\subsection{Cím}

\subsubsection{Alcím}




\end{document}